\documentclass[12pt,letterpaper]{article}

\usepackage[brazilian]{babel}
\usepackage[utf8]{inputenc}
\usepackage[T1]{fontenc}

\usepackage{fullpage}
\usepackage[top=2cm, bottom=4.5cm, left=2.5cm, right=2.5cm]{geometry}
\usepackage{amsmath,amsthm,amsfonts,amssymb,amscd}
\usepackage{lastpage}
\usepackage{enumerate}
\usepackage{fancyhdr}
\usepackage{mathrsfs}
\usepackage{xcolor}
\usepackage{graphicx}
\usepackage{listings}
\usepackage{hyperref}

\hypersetup{%
  colorlinks=true,
  linkcolor=blue,
  linkbordercolor={0 0 1}
}
 
\renewcommand\lstlistingname{Algorithm}
\renewcommand\lstlistlistingname{Algorithms}
\def\lstlistingautorefname{Alg.}

\lstdefinestyle{Python}{
    language        = Python,
    frame           = lines, 
    basicstyle      = \footnotesize,
    keywordstyle    = \color{blue},
    stringstyle     = \color{green},
    commentstyle    = \color{red}\ttfamily
}

\setlength{\parindent}{0.0in}
\setlength{\parskip}{0.05in}

% Edit these as appropriate
\newcommand\course{Lucas Moschen}
\newcommand\hwnumber{1}                  % <-- homework number
\newcommand\NetIDa{netid19823}           % <-- NetID of person #1
\newcommand\NetIDb{netid12038}           % <-- NetID of person #2 (Comment this line out for problem sets)

\pagestyle{fancyplain}
\headheight 35pt              % <-- Comment this line out for problem sets (make sure you are person #1)
\chead{\textbf{\Large Monitorias 10 e 11}}
\rhead{\course \\ \today}
\lfoot{}
\cfoot{}
\rfoot{\small\thepage}
\headsep 1.5em

\begin{document}

\section*{Definições e Teoremas}
\textbf{Lembrando: } Uma transformação linear $T: \mathbb{R}^n \to \mathbb{R}^m$ fica inteiramente determinada por uma matriz $A = [a_{ij}] \in M(m \times n)$. Os vetores coluna dessa matriz são as imagens $A \cdot e_j$ dos vetores da base canônica. Definimos $A$ como matriz de transformação. Assim, $A \cdot e_j = \sum_{i=1}^m a_{ij}e_i (j=1,...,n)$, onde $e_i \in \mathbb{R}^m$.

\textbf{Simetrias: } Matrizes de tranformação referentes à simetria em relação aos eixos x e y, e em relação à origem, respectivamente:
$$
S_x = \left[
\begin{array}{cc}
1 & 0  \\
0 & -1
\end{array}
\right]
S_y = \left[
\begin{array}{cc}
-1 & 0  \\
0 & 1
\end{array}
\right]
S_o = \left[
\begin{array}{cc}
-1 & 0  \\
0 & -1
\end{array}
\right]
$$

\textbf{Dilatações: } Basta multiplicar uma coluna que se quer dilatar por $r$. Podemos chamar $r$ de coeficiente de dilatação. 

\textbf{Rotação: }Para montar essa matriz, basta conhecer a transformação dos vetores $(1,0)$ e $(0,1)$.
$$
R_{\theta} = \left[
\begin{array}{cc}
\cos \theta & - \sin \theta  \\
\sin \theta & \cos \theta
\end{array}
\right]
$$
A rotação tem algumas propriedades:
\begin{itemize}
    \item $R_{\theta}^{-1} = R_{-\theta}$
    \item $R_{\alpha}R_{\beta} = R_{\alpha + \beta}$
    \item $(R_{\theta})^n = R_{n\theta}$
\end{itemize}

\textbf{Projeções:} Podemos considerar a transformação que projeta os vetores sobre a reta $y = ax$. 
$$
P = \frac{1}{1+a^2} \left[
\begin{array}{cc}
1 & a  \\
a & a^2
\end{array}
\right]
$$
Se quisermos que a projeção sobre um eixo e paralelo a uma reta, temos que
$$
P_p = \left[
\begin{array}{cc}
1 & -\frac{1}{a}  \\
0 & 0
\end{array}
\right]
$$

\textbf{Núcleo de $A$: }$N(A) = \{v \in E | Av = 0\}$. É o espaço anulado da matriz $A$. \\
\textbf{Imagem ed $A$:} $Im(A) = \{Av | v \in E\} \implies \exists v \in E; Av = w \implies w \in Im(A) $.
Notemos que $posto(A) = dim~Im(A) = dim~col(A)$. Isto ocorre, pois $w \in Im(A)$ é combinação linear das colunas da matriz $A$. \\
\textbf{Transformação Injetiva:} $A: E \to F$ é injetiva se $\forall v, v', v \neq v' \implies Av \neq Av'$. Uma transformação é injetiva se, e só se, transforma vetores LI em vetores LI. Para essa demonstração, é necessário mostrar que uma transformação é injetiva se, e só se, seu núcleo possui apenas o vetor nulo. \\
\textbf{Transformação Sobrejetiva: }Ocorre quando $Im(A) = F$, onde $F$ é o espaço vetorial contradomínio. \\
\textbf{Teorema do Núcleo e da Imagem: } Como $dim~Im(A) = posto(A)$, podemos usar no teorema do posto. Podemos alterar $n$ para $dim~E$, sendo $E$ o domínio da transformação.
\textbf{Laplace: }Escolhe-se uma linha uma coluna e para cada elemento, calcula-se o seu cofator. $A_{ij} = (-1)^{i+j}D_{ij}$.
\textbf{Propriedades Importantes: } $det(A) = det(A^{T})$; trocar duas linhas ou colunas inverte o sinal do determinante; duas linhas proporcionais indica determinante 0; multiplicar uma linha por $\alpha$ implicará multiplicar o determinante pelo mesmo fator; determinante do produto de matrizes é o produto dos determinantes;  o determinante de uma matriz com a operação de somar com múltiplo de outra linha é idêntico; determinante da inversa é o inverso do determinante 

\section*{Lembretes para exercícios: }
\begin{enumerate}
    \item Para calcular uma matriz de tranformação, precisamos apenas saber a transformação linear de uma base do domínio. Com essa transformação, precisamos obter a transformação da base canônica, para que a matriz seja constrída nessa base. Essa matriz de tranformação também pode ser obtida por $T = AP^{-1}$, onde $P$ tem como colunas os vetores da base, e $A$ os vetores da base após a transformação.
    \item Para mostrar injetividade, podemos usar a contrapositiva da definição. 
    \item Você sabe encontrar uma base para o núcleo e uma base para a imagem de uma transformação? A base da imagem é basicamente a base para o espaço coluna (consegue enxergar o porquê? Tente representar um vetor da imagem como combinação linear das colunas. E a base para o núcleo?  
\end{enumerate}

\section*{Exercícios:}

\begin{enumerate}
    \item \textbf{Reflexão em torno de uma reta:} Seja $S: \mathbb{R}^2 \to \mathbb{R}^2$ a transformação que reflete um veotr em torno da reta $y = ax$. Assim, a reta é a bissetriz do ângulo entre $v$ e $Sv$ e é perpendicular à reta que liga $v$ a $Sv$.
    \\
    \textbf{Solução: } Seja $P$ a matriz de projeção. Projetamos ortogonalmente $v$ sobre a reta $y = ax$. Assim, teremos que $v + Sv = 2Pv \implies I + S = 2P \implies S = 2P - I$. Outra forma é fazer as tranformações dos vetores da base canônica. 
    
    \item Considere 5 lâmpadas, cada uma com um botão. Cada botão muda o estado da lâmpada e das vizinhas. Todas estão apagadas. Como deixar a primeira, terceira e quinta acesas. 
 
    \item Encontre os números $a,b,c,d$ de modo que o operador $A: \mathbb{R}^2 \to \mathbb{R}^2$, dado por $A(x,y) = (ax + by, cx + dy)$ tenha como núcleo a reta $y = 3x$. 
    
    \item A transformação $A: \mathbb{R} \to \mathbb{R}^n; A(x) = (x,2x,...,nx)$ é uma transformação injetiva? E $B(x,y) = (x + 2y, x + y, x - y)$?  
    
    \item Considere uma transformação $A: E \to F$ na base canônica. Considere $V$ uma base de vetores de $E$. Determine a matriz de transformação $A'$ nessa base. Ou seja, se $Av = w \to A'v_V = w_V$. 
    \item Ache uma transformação $A: \mathbb{R}^2 \to \mathbb{R}^2$ tal que a imagem e o núcleo sejam o eixo x.
\end{enumerate}



\end{document}
