\documentclass[a4paper,12pt]{article}

%%%%%%%%%%%%%%%%%%%%%%%%%%%%%%%%%%%%%%%%%%%%%%%%
% Packages
%%%%%%%%%%%%%%%%%%%%%%%%%%%%%%%%%%%%%%%%%%%%%%%%

\usepackage[right=2.5cm, left=2.5cm, top=2.5cm, bottom=2.5cm]{geometry} 
\usepackage[portuguese]{babel}
\usepackage[T1]{fontenc}
\usepackage[utf8]{inputenc}
\usepackage{url}
\usepackage{hyperref}
\Urlmuskip=0mu  plus 10mu

% no indentation
%\usepackage{setspace}
%\setlength{\parindent}{0in}

\usepackage{graphicx} 
\usepackage{float}
\usepackage{xcolor}

\usepackage{mathtools}
\usepackage{amssymb, amsthm}

% headers
\usepackage{fancyhdr}

%%%%%%%%%%%%%%%%%%%%%%%%%%%%%%%%%%%%%%%%%%%%%%%%
% Proper definitions
%%%%%%%%%%%%%%%%%%%%%%%%%%%%%%%%%%%%%%%%%%%%%%%%
\newcommand{\R}{\mathbb{R}}
\newcommand{\B}{\mathcal{B}}

\newtheoremstyle{exer}{}{}{\color{blue}}{}{\color{blue}\bfseries}{}{ }{}
\theoremstyle{exer}
\newtheorem{exercise}{Exercício}

\theoremstyle{definition}
\newtheorem{solution}{Solução}
\newtheorem{definition}{Definição}

\theoremstyle{plain}
\newtheorem{remark}{Observação}



%%%%%%%%%%%%%%%%%%%%%%%%%%%%%%%%%%%%%%%%%%%%%%%%
% Header (and Footer)
%%%%%%%%%%%%%%%%%%%%%%%%%%%%%%%%%%%%%%%%%%%%%%%%

\pagestyle{fancy} 
\fancyhf{}

\lhead{\footnotesize CS: Lista 6}
\rhead{\footnotesize Prof. Asla e Mon. Lucas} 
\cfoot{\footnotesize \thepage} 


\begin{document}

%%%%%%%%%%%%%%%%%%%%%%%%%%%%%%%%%%%%%%%%%%%%%%%%
% Title section of the document
%%%%%%%%%%%%%%%%%%%%%%%%%%%%%%%%%%%%%%%%%%%%%%%%

\thispagestyle{empty} 

\begin{tabular*}{0.95\textwidth}{l @{\extracolsep{\fill}} r} 
    {\large \bf Curvas e Superfícies 2021.1} &  \\
    Escola de Matemática Aplicada, Fundação Getulio Vargas &  \\
    Professora Asla Medeiros e Sá &  \\ 
    Monitor Lucas Machado Moschen & Entrega 26/05/2021\\
    \hline \\
\end{tabular*} 
\vspace*{0.3cm} 

\begin{center}
	{\Large \bf Lista 6}
	\vspace{2mm}
\end{center}  
\vspace{0.4cm}

\begin{exercise}
    Provar que toda bola aberta $\B(x; r)$ é um conjunto aberto.
\end{exercise}

\begin{solution}
    Seja $y \in \B(r; x)$. Queremos provar que existe $\epsilon > 0$ tal que
    $\B(y; \epsilon) \subseteq \B(r; x)$. Definimos para isto $\epsilon := r -
    |y - x| > 0$. Logo, dado qualquer ponto $z \in \B(y; \epsilon)$, temos que
    $$
    |z - x| \le |z - y| + |y - x| < \epsilon + |y - x| = r - |y - x| + |y - x| = r.
    $$
    Logo $z \in \B(x; r)$. Isto é, $\B(y; \epsilon) \subseteq \B(x; r)$. Concluímos que $\B(x; r)$ é aberto.
\end{solution}

\begin{exercise}
    Provar que $Z := \{(x, y) \in \R^2 : xy < 0\}$ é aberto. Dica: Seja $(a, b)$ no conjunto $Z$. Seja
    $\epsilon := \min\{|a|, |b|\} > 0$. Provar que $\B((a, b); \epsilon) \subseteq Z$.
\end{exercise}

\begin{solution}

\end{solution}

\begin{exercise}
    Provar que união de conjuntos abertos é um conjunto aberto.
\end{exercise}

\begin{solution}
    Seja $\{A_{\lambda} : \lambda \in \Lambda\}$ uma família de abertos, onde
    $\Lambda$ é um conjunto de índices (possívelmente infinito, não
    enumerável). Consideremos a união:
    $$
    A := \bigcup_{\lambda \in \Lambda} A_{\lambda}.
    $$
    Seja $z \in A$. Logo $z \in A_{\lambda}$ para algum índice $\lambda$. Dado
    que $A_{\lambda}$ é aberto, existe $\epsilon > 0$ tal que $\B(z; \epsilon)
    \subseteq A_{\lambda}$. Logo $\B(z; \epsilon) \subseteq A$. Concluímos que $A$ é aberto.
\end{solution}

\begin{exercise}
    Provar que a interseção de uma quantidade finita de abertos é um conjunto aberto.
\end{exercise}

\begin{solution}

\end{solution}

\begin{exercise}
    Provar que a interseção de conjuntos fechados é um conjunto fechado. Será que união de fechados é também fechado? Se não for certo, dar um contraexemplo.
\end{exercise}

\begin{solution}

\end{solution}

\begin{exercise}
    Dê exemplos de conjuntos que não são nem abertos nem fechados.
\end{exercise}

\begin{solution}

\end{solution}

\begin{exercise}
    Prove que 
    $$C = \{(x, y) \in \R^2 : y > 0\}$$
    é aberto.
\end{exercise}

\begin{solution}

\end{solution}

\begin{exercise}
    Prove que um conjunto em $\R^n$ é aberto se, e somente se, é união de bolas abertas.
\end{exercise}

\begin{solution}

\end{solution}

\begin{exercise}
    Provar que $\R \times \{0\}$ é fechado em $\R^2$.
\end{exercise}

\begin{solution}

\end{solution}

\begin{exercise}
    Prove que as bolas fechadas são conjuntos fechados.
\end{exercise}

\begin{solution}

\end{solution}

\begin{exercise}
    Seja $A \subseteq \R^n$ tal que existe $d > 0$ tal que $||x - y|| \ge d$
    para todo par de pontos $x, y \in A$. Prove que $A$ é fechado em $\R^n$.
\end{exercise}

\begin{solution}
  
\end{solution}

\begin{exercise}
    Seja $A \subseteq \R^2$ um conjunto não vazio contido numa reta de $\R^2$. Prove que $A$ não é
    aberto.
\end{exercise}

\begin{solution}
    
\end{solution}

\begin{exercise}
    Seja $A \subseteq \R^n$. Prove que $\R^n/int(A)$ é fechado.
\end{exercise}

\begin{solution}
 
\end{solution}

\begin{exercise}
    Seja $A \subset B \subseteq \R^n$, e $x$ ponto de acumulação de $A$. Será que $x$ é também ponto de
    acumulação de $B$?
\end{exercise}

\begin{solution}
 
\end{solution}

\begin{exercise}
    Se $A \subset \R^n$ é aberto, prove que sua fronteira tem interior vazio.
\end{exercise}

\begin{solution}

\end{solution}

\begin{exercise}
    Seja $A \subseteq \R^n$ com $n \ge 2$. Prove que, dado $a \in \R^n/A$, o
    conjunto $A \cup \{a\}$ é aberto se, e somente se, $a$ é um ponto isolado da
    fronteira de $A$.
\end{exercise}

\begin{solution}
    
\end{solution}

\begin{exercise}
    Prove que se $F \subseteq \R^n$ é fechado então sua fronteira tem interior vazio.
\end{exercise}

\begin{solution}
    
\end{solution}

\begin{exercise}
    Sejam $F \in \R^n$ fechado e $f : F \to \R^m$ uma aplicação contínua.
    Mostre que $f$ leva subconjuntos limitados de $F$ em subconjuntos
    limitados de $\R^m$. Prove, exibindo um contra-exemplo, que não se conclui o mesmo removendo-se a hipótese de $F$ ser fechado.
\end{exercise}

\begin{solution}
    
\end{solution}

\begin{exercise}
    Prove que duas bolas abertas de $\R^n$ são homeomorfas.
\end{exercise}

\begin{solution}
    Dados $a \in \R^n$ e $r > 0$, consideremos a aplicação:
    \begin{align*}
        f : \B(0, 1) &\to \B(a, r)\\
        x &\mapsto rx + a        
    \end{align*}
    A aplicação $f$ é bijetiva e contínua. Sua inversa, $f^{-1} : \B(a, r) \to
    \B(0, 1)$, é dada por $f^{-1}(y) = \frac{1}{r}(y - a)$, donde se vê que
    $f^{-1}$ é contínua, portanto $f$ é um homeomorfismo. Pela transitividade
    da relação de homeomorfismo, conclui-se que duas bolas abertas quaisquer de
    $\R^n$ são homeomorfas. Um argumento análogo prova que vale o mesmo para duas bolas, ambas, fechadas.
\end{solution}

\begin{exercise}
    Verifique que a aplicação:
    \begin{align*}
        f : \B(0, 1) &\to \R^n \\
        x &\mapsto \frac{x}{1 - ||x||}        
    \end{align*}
    é um homeomorfismo entre a bola aberta unitária $\B(0, 1)$ e $\R^n$.
    Conclua que qualquer bola aberta de $\R^n$ é homeomorfa a todo o espaço $\R^n$.
\end{exercise}

\begin{solution}
    
\end{solution}

\begin{exercise}
    Mostre que o cone $C = \{(x, y, z) \in \R^3 ; z = x^2 + y^2 \}$ e $\R^2$ são homeomorfos.
\end{exercise}

\begin{solution}
    
\end{solution}

% \begin{thebibliography}{9}
%     \bibitem{pressley} 
%     Pressley, Andrew N. Elementary differential geometry. Springer Science \& Business Media, 2010.
% \end{thebibliography}

\end{document}