\documentclass[a4paper,12pt]{article}

%%%%%%%%%%%%%%%%%%%%%%%%%%%%%%%%%%%%%%%%%%%%%%%%
% Packages
%%%%%%%%%%%%%%%%%%%%%%%%%%%%%%%%%%%%%%%%%%%%%%%%

\usepackage[right=2.5cm, left=2.5cm, top=2.5cm, bottom=2.5cm]{geometry} 
\usepackage[portuguese]{babel}
\usepackage[T1]{fontenc}
\usepackage[utf8]{inputenc}
\usepackage{url}
\usepackage{hyperref}
\Urlmuskip=0mu  plus 10mu

% no indentation
%\usepackage{setspace}
%\setlength{\parindent}{0in}

\usepackage{graphicx} 
\usepackage{float}
\usepackage{xcolor}

\usepackage{mathtools}
\usepackage{amssymb, amsthm}

% headers
\usepackage{fancyhdr}

%%%%%%%%%%%%%%%%%%%%%%%%%%%%%%%%%%%%%%%%%%%%%%%%
% Proper definitions
%%%%%%%%%%%%%%%%%%%%%%%%%%%%%%%%%%%%%%%%%%%%%%%%
\newcommand{\R}{\mathbb{R}}

\newtheoremstyle{exer}{}{}{\color{blue}}{}{\color{blue}\bfseries}{}{ }{}
\theoremstyle{exer}
\newtheorem{exercise}{Exercício}

\theoremstyle{definition}
\newtheorem{solution}{Solução}
\newtheorem{definition}{Definição}

\theoremstyle{plain}
\newtheorem{remark}{Observação}



%%%%%%%%%%%%%%%%%%%%%%%%%%%%%%%%%%%%%%%%%%%%%%%%
% Header (and Footer)
%%%%%%%%%%%%%%%%%%%%%%%%%%%%%%%%%%%%%%%%%%%%%%%%

\pagestyle{fancy} 
\fancyhf{}

\lhead{\footnotesize CS: Lista 5}
\rhead{\footnotesize Prof. Asla e Mon. Lucas} 
\cfoot{\footnotesize \thepage} 


\begin{document}

%%%%%%%%%%%%%%%%%%%%%%%%%%%%%%%%%%%%%%%%%%%%%%%%
% Title section of the document
%%%%%%%%%%%%%%%%%%%%%%%%%%%%%%%%%%%%%%%%%%%%%%%%

\thispagestyle{empty} 

\begin{tabular*}{0.95\textwidth}{l @{\extracolsep{\fill}} r} 
    {\large \bf Curvas e Superfícies 2021.1} &  \\
    Escola de Matemática Aplicada, Fundação Getulio Vargas &  \\
    Professora Asla Medeiros e Sá &  \\ 
    Monitor Lucas Machado Moschen & Entrega 13/05/2021\\
    \hline \\
\end{tabular*} 
\vspace*{0.3cm} 

\begin{center}
	{\Large \bf Lista 5}
	\vspace{2mm}
\end{center}  
\vspace{0.4cm}

\begin{exercise}
    Seja $F : \R^2 \to \R^3$ uma aplicação linear. Mostre que: $F$ é injetora
    se, e só se, a imagem da base canônica de $\R^2$ forma um conjunto de
    vetores linearmente independentes de $\R^3$ ou, equivalentemente, se a
    matriz associada de $F$ tem posto 2. (obs.: Repare que este resultado está
    sendo usado para o conceito de superfície regular descrito acima).
\end{exercise}

\begin{solution}
    Suponha que $F$ é injetora. Seja $\{e_1, e_2\}$ a base canônica de $\R^2$.
    Suponha que $F(e_1) = \alpha F(e_2)$ para algum $\alpha \in \R$. Pela
    linearidade de $F$, 
    $$F(e_1) = F(\alpha e_2) \implies (1,0) = e_1 = \alpha e_2 = (0,\alpha),$$
    um absurdo. Concluímos que $\{F(e_1), F(e_2)\}$ é um
    conjunto linearmente independente.   

    Agora suponha que $F(e_1)$ e $F(e_2)$ formam um conjunto linearmente
    independente. Sejam $x = \alpha_1 e_1 + \alpha_2 e_2$ e $y = \beta_1 e_1 +
    \beta_2 e_2$ de forma que $F(x) = F(y)$. Assim 
    $$
    F(\alpha_1 e_1 + \alpha_2 e_2) = F(\beta_1 e_1 + \beta_2 e_2) \implies (\alpha_1 - \beta_1)F(e_1) + (\alpha_2 - \beta_2)F(e_2) = 0.
    $$
    Pela independência linear, sabemos que essa última igualdade implica 
    $$
    \alpha_1 - \beta_1 = \alpha_2 - \beta_2 = 0 \implies \alpha_1 = \beta_1 \text{ e } \alpha_2 = \beta_2 \implies x = y
    $$
    e, portanto, $F$ é injetora. 
\end{solution}

\begin{exercise}
    Mostre que o paraboloide hiperbólico $S = \{(x, y, z) \in \R^3; z = x^2 -
    y^2\}$ é uma superfície regular. Desenhe o paraboloide em um ambiente
    gráfico juntamente com o plano tangente e um vetor normal à superfície.
    Faça o desenho de forma a poder variar o ponto aonde o plano tangente é exibido.
\end{exercise}

\begin{definition}
    Dizemos que $S \subseteq \mathbb{R}^n$ é uma {\em superfície} se para todo
    ponto $p \in S$, existe um conjunto aberto $U \subset \mathbb{R}^2$ e um
    conjunto aberto $W \subseteq \mathbb{R}^3$ contendo $p$ tal que $S \cap W$
    é homeomorfo a $U$.  O homeomorfismo $\sigma : U \to S \cap W$ é um
    {\em patch} ou {\em parametrização}. A coleção desses homeomorfismos que cobrem
    $S$ é chamada de {\em atlas}. 
\end{definition}

\begin{solution}
    Para mostrar que um conjunto é uma superfície, podemos, inicialmente,
    propor uma parametrização. Nesse caso, a parametrização natural é 
    $$
    \sigma((x,y)) = (x,y,x^2 - y^2)
    $$
    Tome $p = (x,y,x^2 - y^2)\in S$. Defina $U = \R^2$ e $p \in W = \R^3 $ que são
    abertos. Tenho que mostrar que $\sigma : U \to S \cap W$ como descrita a
    cima é um homeomorfismo. 

    \begin{enumerate}
        \item[(i)] Injetividade: De fato, se $\sigma((x_1, y_1)) =
        \sigma((x_2, y_2))$, então é claro que $x_1 = x_2$ e $y_1 = y_2$ pela
        própria definição de $\sigma$. 
        \item[(ii)] Sobrejetividade: Seja $(x,y,x^2 - y^2) \in W \cap S = S$.
        Está claro que $(x,y)$ mapeia nesse valor. 
        \item[(iii)] Continuidade: Está claro a continuidade e a continuidade
        da inversa (que é projeção do parabolóide sobre o plano).
    \end{enumerate}

    Isso prova que $\sigma$ é um patch e $S$ é uma superfície. Além disso,
    vemos que $\sigma$ é uma função suave com 
    $$
    \sigma_x = (1, 0, 2x), \sigma_y = (0,1,-2y)
    $$
    que são linearmente independentes. Portanto $S$ é uma superfície regular. 
\end{solution}

\begin{exercise}
    Mostre que, se $f(u, v)$ é uma função real diferenciável, onde $(u, v) \in
    U$, aberto de $\R^2$, então a aplicação $X(u, v) = (u, v, f(u, v))$ é uma
    superfície parametrizada regular, que descreve o gráfico da função $f$.
\end{exercise}

\begin{solution}
    Observe que esse exercício é uma generalização do exercício 2. Basta
    tomarmos $W = \R^3$ e verificarmos que $X(u,v)$ é um homeomorfismo. 

    \begin{enumerate}
        \item[(i)] Injetividade: deriva da mesma demonstração do exercício
        anterior. 
        \item[(ii)] Sobrejetividade: basta fazer a projeção do ponto
        $(u,v,f(u,v))$ sobre $U$ que visualizamos a sobrejetividade. 
        \item[(iii)] Contínuidade de $X$: tome uma sequência $\{u_n, v_n\}_{n \in
        \mathbb{N}}$ convergente para $(u,v)$. Pela continuidade de $f$,
        teremos que $\forall \epsilon > 0$, 
        $$
        ||(f(u_n, v_n) - f(u,v), v_n - v, u_n - u|| \le ||f(u_n, v_n) - f(u,v)|| + ||(v_n - v, u_n - v)|| \to 0
        $$
        \item[(iv)] Continuidade da inversa: a inversa é a projeção sobre o
        plano, que é contínua. 
    \end{enumerate}
    Logo o conjunto em $\R^3$ gerado por $X$ é uma superfície. Podemos
    verificar a diferenciabilidade da mesma forma que a continuidade. Ela saí
    diretamente da diferenciabilidade de $f$. Por fim 
    $X_u = (1,0,f_u)$ e $X_v = (0,1,f_v)$ são linearmente independentes, o que
    mostra que a superfície é regular. 
\end{solution}

\begin{exercise}
    Considere o hiperbolóide de uma folha
    $$
    S := \{(x, y, z) \in \R^3 : x^2 + y^2 - z^2 = 1\}
    $$
    Mostre que, para todo $\theta$, a reta
    $$
    (x - z)\cos(\theta) = (1 - y)\sin(\theta), (x + z)\sin(\theta) = (1 + y)\cos(\theta)
    $$
    está contida em $S$, e que, todo ponto do hiperboloide está em alguma
    dessas linhas. Desenhe o hiperbolóide e as linhas em um ambiente gráfico.
    Deduza que a superfície pode ser coberta por uma única parametrização.
\end{exercise}

\begin{solution}
    Seja $r_{\theta}$ a reta determinada pelas equações
    \begin{equation}
        \label{eq-1}
        (x - z)\cos(\theta) = (1 - y)\sin(\theta)
    \end{equation}
    e 
    \begin{equation}
        \label{eq-2}
        (x + z)\sin(\theta) = (1 + y)\cos(\theta)
    \end{equation}  
    para $\theta$ fixo. Tome $p \in r_{\theta}$. Assim, vale que 
    \begin{equation}
        \label{eq-3}
        (x - z)(x + z)\cos(\theta)\sin(\theta) = (1-y)(1+y)\cos(\theta)\sin(\theta)
    \end{equation}
    Se $\theta = k\pi, k \in \mathbb{Z}$, pela equação \ref{eq-1}, $x = z$ e
    equação \ref{eq-2}, $y=-1$. Logo $x^2 + y^2 -z^2 = y^2 = 1$ e, portanto,
    $p \in S$. Se $\theta = \pi/2 + k\pi, k \in \mathbb{Z}$, por \ref{eq-1},
    $y=1$ e por \ref{eq-2}, $x = -z$ e pela mesma conta, $p \in S$. 
    
    Considere $\theta$ diferente desses valores. Assim,
    $\cos(\theta)\sin(\theta) \neq 0$ e, portanto, 
    $$
    (x-z)(x+z) = (1 - y)(1 + y) \implies x^2 - z^2 = 1 - y^2 \implies x^2 + y^2 - z^2 = 1
    $$
    e, portanto, $p \in S$. Provamos, portanto, que para todo $\theta \in \R,
    r_{\theta} \subset S$. 

    Tome agora $p = (x,y,z) \in S$, isto é, $x^2 + y^2 - z^2 = 1$ o que
    implica
    \begin{equation}
        \label{eq-4}
        (x + z)(x - z) = (1 + y)(1 - y).
    \end{equation}
    Podemos parametrizar através de coordenadas polares:  
    \begin{align*}
        x + z = r\cos(\theta), \\
        1 + y = r\sin(\theta).
    \end{align*}
    Se $x + z \neq 0$ ou $1 + y \neq 0$, então $r \neq 0$ e $(x -
    z)\sin(\theta) = (1 + y)\cos(\theta)$ pela equação \ref{eq-4}. Se $x + y =
    0$ e $1 + y = 0$, de fato teremos $(x + z)\sin(\theta) = (1 +
    y)\cos(\theta)$. Além disso:
    \begin{align*}
        1 - y = r_2\cos(\theta), \\
        x - z = r_2\sin(\theta).
    \end{align*}
    Para respeitar \ref{eq-4}, $\theta$ não pode ser diferente. 
    Se $x - z \neq 0$ ou $1 - y \neq 0$, então $r \neq 0$ e $(x +
    z)\sin(\theta) = (1 + y)\cos(\theta)$. Se $x + z = 0$  e $1 + y = 0$, 
    $$
    (x-z)\cos(\theta) = r_2\sin(\theta)\cos(\theta) = (1-y)\sin(\theta).
    $$
    Se $x - z = 0$ e $1 - y = 0$, 
    $$
    (x+z)\sin(\theta) = r\sin(\theta)\cos(\theta) = (1+y)\cos(\theta).
    $$   
    Concluo que $p \in r_{\theta}$ e $S \subset \cup_{\theta} r_{\theta}$. 

    Note que os vetores normais aos planos são, respectivamente,
    $(\cos(\theta), \sin(\theta), -\cos(\theta))$ e $(\sin(\theta),
    -\cos(\theta), \sin(\theta))$. O vetor diretor à reta $r_{\theta}$ é
    qualquer vetor ortogonal a esses dois. Em particular, 
    $$
    (\cos(\theta), \sin(\theta), -\cos(\theta)) \times (\sin(\theta),
    -\cos(\theta), \sin(\theta)) = -(\cos(2\theta), \sin(2\theta), 1) 
    $$
    Por simplicidade, usaremos o vetor $(\cos(2\theta), \sin(2\theta), 1)$.
    Além disso, essa reta, quando $z = 0$ passa pelo ponto solução de:
    $$
    \begin{cases}
      x\cos(\theta) + y\sin(\theta) = \sin(\theta) \\
      x\sin(\theta) - y\cos(\theta) = \cos(\theta), 
    \end{cases}
    $$
    que é $x = \sin(2\theta)$ e $y = \cos(2\theta)$. Assim, a reta pode ser
    escrita como 
    $$
    r_{\theta}(t) = (\sin(2\theta), -\cos(2\theta), 0) + t(\cos(2\theta), \sin(2\theta), 1)
    $$
    Defina $\sigma : \R^2 \to \R^3$ como $\sigma(\theta, t) = r_{\theta}(t)$.
    Tomando $\sigma$ restrito a $(0,\pi)$ e depois restrito a $(-\pi/2,
    \pi/2)$, teremos que $S$ é uma superfície. 
\end{solution}

\begin{exercise}
    Considere uma curva regular $\alpha(s) = (x(s), y(s), z(s)), s \in I
    \subset \R$. Seja o subconjunto de $\R^3$ gerado pelas retas que passam
    por $\alpha(s)$, paralelas ao eixo $O_z$. Dê uma condição suficiente que
    deve satisfazer a curva $\alpha$ para que $S$ seja o traço de uma
    superfície parametrizada regular.
\end{exercise}

\begin{solution}
    A superfície gerada é um exemplo de superfície regrada \cite[Seção
    5.3]{pressley}. Seja um ponto $p$ dessa superfície. Sabemos que $p$ está
    em uma dessas retas paralelas ao eixo $O_z$, isto é, geradas pelo vetor
    $(0,0,1)$. Essa reta atinge a curva $\alpha$ em $u \in I$. Assim, 
    $$
    p = \alpha(u) + v(0,0,1) = (x(u), y(u), z(u) + v).
    $$
    Defina $\sigma : U \to \R^3$ em $U = \{(u,v) \in \R^2 | u \in I\}$ como 
    $$
    \sigma(u,v) = (x(u), y(u), z(u) + v).
    $$
    Pela regularidade de $\alpha$, sabemos que $\sigma$ é diferenciável. Além
    disso, 
    $$
    \sigma_u = (x_u, y_u, z_u) \neq 0
    $$
    e
    $$
    \sigma_v = (0,0,1)
    $$
    de forma que 
    $$
    \sigma_u \times \sigma_v = (y_u, -x_u, 0)
    $$
    só é nulo quando $y_u = x_u = 0$. Desta forma, a superfície é regular se, e
    somente se, $\alpha$ é nunca tangente ao vetor $(0,0,1)$. 

\end{solution}

\begin{exercise}
    {\bf Extra:} Mostre que o cilindro circular
    $$
    S := \{(x, y, z) \in \R^3 : x^2 + y^2 = 1\}
    $$
    pode ser descrito por uma parametrização global, isto é, que existe um
    atlas composto só por uma única carta.
\end{exercise}

\begin{solution}
    \url{https://math.stackexchange.com/questions/1664320/showing-a-circular-cylinder-is-a-surface}
\end{solution}


\begin{thebibliography}{9}
    \bibitem{pressley} 
    Pressley, Andrew N. Elementary differential geometry. Springer Science \& Business Media, 2010.
\end{thebibliography}

\end{document}