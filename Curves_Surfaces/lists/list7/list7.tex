\documentclass[a4paper,12pt]{article}

%%%%%%%%%%%%%%%%%%%%%%%%%%%%%%%%%%%%%%%%%%%%%%%%
% Packages
%%%%%%%%%%%%%%%%%%%%%%%%%%%%%%%%%%%%%%%%%%%%%%%%

\usepackage[right=2.5cm, left=2.5cm, top=2.5cm, bottom=2.5cm]{geometry} 
\usepackage[portuguese]{babel}
\usepackage[T1]{fontenc}
\usepackage[utf8]{inputenc}
\usepackage{url}
\usepackage{hyperref}
\Urlmuskip=0mu  plus 10mu

% no indentation
%\usepackage{setspace}
%\setlength{\parindent}{0in}

\usepackage{graphicx} 
\usepackage{float}
\usepackage{xcolor}

\usepackage{mathtools}
\usepackage{amssymb, amsthm}
\usepackage{cancel}

% headers
\usepackage{fancyhdr}

%%%%%%%%%%%%%%%%%%%%%%%%%%%%%%%%%%%%%%%%%%%%%%%%
% Proper definitions
%%%%%%%%%%%%%%%%%%%%%%%%%%%%%%%%%%%%%%%%%%%%%%%%
\newcommand{\R}{\mathbb{R}}
\newcommand{\B}{\mathcal{B}}
\newcommand{\sur}{\mathcal{S}}

\newtheoremstyle{exer}{}{}{\color{blue}}{}{\color{blue}\bfseries}{}{ }{}
\theoremstyle{exer}
\newtheorem{exercise}{Exercício}

\theoremstyle{definition}
\newtheorem{solution}{Solução}
\newtheorem{definition}{Definição}

\theoremstyle{plain}
\newtheorem{remark}{Observação}



%%%%%%%%%%%%%%%%%%%%%%%%%%%%%%%%%%%%%%%%%%%%%%%%
% Header (and Footer)
%%%%%%%%%%%%%%%%%%%%%%%%%%%%%%%%%%%%%%%%%%%%%%%%

\pagestyle{fancy} 
\fancyhf{}

\lhead{\footnotesize CS: Lista 7}
\rhead{\footnotesize Prof. Asla e Mon. Lucas} 
\cfoot{\footnotesize \thepage} 


\begin{document}

%%%%%%%%%%%%%%%%%%%%%%%%%%%%%%%%%%%%%%%%%%%%%%%%
% Title section of the document
%%%%%%%%%%%%%%%%%%%%%%%%%%%%%%%%%%%%%%%%%%%%%%%%

\thispagestyle{empty} 

\begin{tabular*}{0.95\textwidth}{l @{\extracolsep{\fill}} r} 
    {\large \bf Curvas e Superfícies 2021.1} &  \\
    Escola de Matemática Aplicada, Fundação Getulio Vargas &  \\
    Professora Asla Medeiros e Sá &  \\ 
    Monitor Lucas Machado Moschen & Entrega 07/06/2021\\
    \hline \\
\end{tabular*} 
\vspace*{0.3cm} 

\begin{center}
	{\Large \bf Lista 7}
	\vspace{2mm}
\end{center}  
\vspace{0.4cm}

\begin{exercise}[6.1.1]
    Calcule a primeira forma fundamental das seguintes superfícies: 
    \begin{enumerate}
        \item[(i)] $\sigma(u,v) = (\sinh(u)\sinh(v), \sinh(u)\cosh(v), \sinh(u))$.
        \item[(ii)] $\sigma(u,v) = (u-v,u+v,u^2+v^2)$.
        \item[(iii)] $\sigma(u,v) = (\cosh(u),\sinh(u),v)$.
        \item[(iv)] $\sigma(u,v) = (u,v,u^2+v^2)$.
        
        Que tipos de superfícies são estas? 
    \end{enumerate}
\end{exercise}

\begin{solution}
    Qualquer vetor tangente a uma superfície $\sur$ (definida por $\sigma$) no
    ponto $p$ pode ser unicamente escrita como uma combinação linear de
    $\sigma_u$ e $\sigma_v$ (o espaço tangente de uma superfície $\sur$ no ponto
    $p$ é o plano tangente gerado pelo produto vetorial das derivadas parciais
    $\sigma_u \times \sigma_v$). Então fazemos 
    $$
    E = ||\sigma_u||^2, F = \langle \sigma_u, \sigma_v \rangle, G = ||\sigma_v||^2
    $$
    e
    $$
    Edu^2 + 2Fdudv + Gdv^2
    $$
    será a primeira forma fundamental tradicional. Assim:
    \begin{enumerate}
        \item[(i)]  $\sigma_u = (\cosh(u)\sinh(v), \cosh(u)\cosh(v),
        \cosh(u))$ e  
        
        $\sigma_v = (\sinh(u)\cosh(v), \sinh(u)\sinh(v), 0)$. Assim 
        $$
        E = ||\sigma_u||^2 = \cosh^2(u)\sinh^2(v) + \cosh^2(u)\cosh^2(v) + \cosh^2(u) = 2\cosh^2(u)\cosh^2(v),
        $$
        $$
        F = \sigma_u\cdot\sigma_v = 2\cosh(u)\sinh(u)\cosh(v)\sinh(v) = \frac{1}{2}\sinh(2u)\sinh(2v),
        $$
        $$
        G = ||\sigma_v||^2 = \sinh^2(u)\cosh^2(v) + \sinh^2(u)\sinh^2(v) = \sinh^2(u)\cosh(2v).
        $$
        Para mais detalhes, veja relações úteis de funções
        hiperbólicas\footnote{\url{https://en.wikipedia.org/wiki/Hyperbolic_functions\#Useful_relations}}.
        Essa superfície é um {\bf cone quádrico} dado pela equação cartesiana $x^2 -
        y^2 + z^2 = 0$. 
        
        \item[(ii)] $\sigma_u = (1,1,2u)$ e $\sigma_v = (-1,1,2v)$. Assim,
        $$
        E = ||\sigma_u||^2 = 1 + 1 + 4u^2 = 4u^2 + 2,
        $$
        $$
        F = \sigma_u\cdot\sigma_v = -1 + 1 + 4uv = 4uv,
        $$
        $$
        G = ||\sigma_v||^2 = 1 + 1 + 4v^2 = 4v^2 + 2.
        $$
        Essa superfície é uma {\bf paraboloide de revolução}.

        \item[(iii)] $\sigma_u = (\sinh(u), \cosh(u), 0)$ e $\sigma_v =
        (0,0,1)$. Assim, 
        $$
        E = ||\sigma_u||^2 = \sinh^2(u) + \cosh^2(u) = \cosh(2u),
        $$
        $$
        F = \sigma_u\cdot\sigma_v = 0,
        $$
        $$
        G = ||\sigma_v||^2 = 1.
        $$
        Essa superfície é um {\bf cilindro hiperbólico}.

        \item[(iv)] $\sigma_u = (1,0,2u)$ e $\sigma_v = (0,1,2v)$. Assim, 
        $$
        E = ||\sigma_u||^2 = 4u^2 + 1,
        $$
        $$
        F = \sigma_u\cdot\sigma_v = 4uv,
        $$
        $$
        G = ||\sigma_v||^2 = 4v^2 + 1.
        $$
        Essa superfície é um {\bf parabolóide de revolução}. Para mais
        detalhes sobre os tipos de superfície, consulte \cite[Seção 5.2]{pressley}.
    \end{enumerate} 
\end{solution}

\begin{exercise}[6.1.3]
    Seja $Edu^2 + 2Fdu dv + Gdv^2$ a primeira forma fundamental do patch
    $\sigma(u, v)$ da superfície $\mathcal{S}$. Mostre que, se $p$ é um
    ponto da imagem de $\sigma$ e $v$, $w \in T_p\mathcal{S}$, então 
    $$\langle v, w \rangle = Edu(v)du(w) + F(du(v)dv(w) + du(w)dv(v)) + Gdv(v)dv(w).$$
\end{exercise}

\begin{solution}
    Primeiro provamos a relação para uma base de $T_p\sur$, em particular,
    $\{\sigma_u, \sigma_v\}$. Nesse caso $du(\sigma_u) = dv(\sigma_v) = 1$ e
    $du(\sigma_v) = dv(\sigma_u) = 0$ e, portanto, 
    $$
    \langle \sigma_u, \sigma_u \rangle = E, \langle \sigma_u, \sigma_v \rangle = F, \text{ e } \langle \sigma_v, \sigma_v \rangle = G,
    $$
    que são relações verdadeiras. Agora seja $v,w \in T_p\sur$. Escrevemos $v
    = \lambda_1\sigma_u + \lambda_2\sigma_v$ e $w = \mu_1\sigma_u +
    \mu_2\sigma_v$. Usamos a linearidade dos mapas $du$ e $dv$, e da
    bilinearidade do produto interno para ver que a relação é válida. 
\end{solution}

\begin{exercise}[6.1.5]
    Mostre que as seguintes condições são equivalentes em um patch $\sigma(u,
    v)$ com primeira forma fundamental $Edu^2 + 2F dudv + Gdv^2$:
    \begin{enumerate}
        \item[(i)] $E_v = G_u = 0$.
        \item[(ii)] $\sigma_{uv}$ é paralelo ao vetor normal padrão $N$. 
        \item[(iii)] Os lados opostos de qualquer quadrilátero formado por curvas
        parâmetros de $\sigma$ tem o mesmo comprimento (veja as observações após a Proposição 4.4.2).   
    \end{enumerate}
    Quando essas condições são satisfeitas, as curvas parâmetros de $\sigma$
    são ditas {\em Chebyshev net.} Mostra que, nesse caso, $\sigma$ tem uma
    parametrização $\tilde{\sigma}(\tilde{u}, \tilde{v})$ com a primeira forma
    fundamental 
    $$
    d\tilde{u}^2 + 2\cos(\theta)d\tilde{u}d\tilde{v} + d\tilde{v}^2,
    $$
    onde $\theta$ é uma função suave de $(\tilde{u}, \tilde{v})$. Mostra que
    $\theta$ é o ângulo entre as curvas parâmetros de $\tilde{\sigma}$.
    Mostre além que, se colocamos $\hat{u} = \tilde{u} + \tilde{v}, \hat{v} =
    \tilde{u} - \tilde{v}$, a reparametrização resultante
    $\hat{\sigma}(\hat{u}, \hat{v})$ de $\tilde{\sigma}(\tilde{u}, \tilde{v})$
    tem primeira forma fundamental 
    $$
    \cos^2(\omega)d\hat{u}^2 + \sin^2(\omega)d\hat{v}^2, 
    $$
    onde $\omega = \theta/2$. 
\end{exercise}

\begin{solution}
    Primeiro vamos provar que (i) é equivalente a (ii). Primeiro, vejamos que 
    $$
    E_v = \frac{d}{dv}\langle \sigma_u, \sigma_u \rangle = \langle \sigma_u, \sigma_{uv} \rangle,
    $$
    $$
    G_u = \frac{d}{du}\langle \sigma_v, \sigma_v \rangle = \langle \sigma_v, \sigma_{uv} \rangle, \text{ e }
    $$
    $$
    N = ||\sigma_u \times \sigma_v||^{-1}(\sigma_u \times \sigma_v).
    $$
    Logo $E_v = G_u = 0$ é equivalente a $\sigma_{uv}$ ser ortogonal a
    $\sigma_u$ e $\sigma_v$, e por conseguinte, paralelo a $N$.  Vamos lembrar
    que as curvas $u \mapsto \sigma(u, v_0)$ e $v \mapsto \sigma(u_0, v)$ para
    $u_0$ e $v_0$ fixados são as curvas parâmetros. Considere o quadrilátero
    determinado pela intersecção das curvas parâmetros determinadas por $u_0,
    u_1, v_0$ e $v_1$. Além disso, quando $u = u_0$, o comprimento é dado por 
    $$
    \sigma_{v_0}^{v_1} ||\sigma_v(u_0, v)|| dv = \int_{v_0}^{v_1} \sqrt{G(u_0, v)}dv.
    $$
    Suponha (i). Quando $G_u = 0$, a função $G$ não varia quando $u$ varia.
    Portanto 
    $$
    \int_{v_0}^{v_1} ||\sigma_v(u_0, v)|| dv = \int_{v_0}^{v_1} \sqrt{G(u_0, v)}dv =
    \int_{v_0}^{v_1} \sqrt{G(u_1, v)}dv = 
    \int_{v_0}^{v_1} ||\sigma_v(u_1, v)|| dv.
    $$
    Como $E_v = 0$, verificamos que os outros dois lados também têm mesmo
    comprimento. Portanto, vale (iii). Agora suponha (iii). Assim 
    $$
    \int_{v_0}^{v_1} \sqrt{G(u_0, v)}dv =
    \int_{v_0}^{v_1} \sqrt{G(u_1, v)}dv = 
    $$
    para quaisquer $u_0$ e $u_1$. Em particular essa integral não depende de
    $u$ e 
    $$
    0 = \frac{d}{du} \int_{v_0}^{v_1} \sqrt{G(s, t)}dt = \int_{v_0}^{v_1} \frac{G_u(s,t)}{\sqrt{G(s,t)}}dt, 
    $$
    para valores $v_0$ e $v_1$ quaisquer. Pela continuidade de $G_u$, se ela
    for não nula em um ponto, ela será não nula em um intervalo $(v_0,v_1)$ e,
    portanto, a integral será também não nula. Logo $G_u = 0$.
    Equivalentemente vemos que $E_v = 0$. 

    Agora, suponhamos as condições acima. Defina $E(u) = E(u,v)$ (pela
    condição (i), $E$ é constante em $v$) e $G(v) = G(u,v)$. Com isso, defina
    $$\tilde{u} = \int \sqrt{E(u)}du, \tilde{v} = \int
    \sqrt{G(v)} dv.$$
    Então o mapa $(u,v) \overset{F}{\mapsto} (\tilde{u}, \tilde{v})$ é uma reparametrização
    com Jacobiano $\sqrt{EG}$ não nulo, portanto invertível \cite[Proposição 4.2.7]{pressley}. A primeira forma fundamental de
    $\tilde{\sigma}(\tilde{u}, \tilde{v}) = \sigma(F^{-1}(\tilde{u},\tilde{v}))$ pode ser escrita como 
    $$
    \tilde{E}d\tilde{u}^2 + 2\tilde{F}d\tilde{u}d\tilde{v} + \tilde{G}d\tilde{v}^2,  
    $$
    em que, pela regra da cadeia, 
    $$
    \tilde{E} = ||\tilde{\sigma}_{\tilde{u}}||^2  = \left|\left|\sigma_u\frac{\partial u}{\partial \tilde{u}} + \sigma_v\cancel{\frac{\partial v}{\partial \tilde{u}}} \right|\right|^2 = E/(\sqrt{E})^2 = 1, 
    $$
    $$
    \tilde{G} = ||\tilde{\sigma}_{\tilde{v}}||^2  = \left|\left|\sigma_u\cancel{\frac{\partial u}{\partial \tilde{v}}} + \sigma_v\frac{\partial v}{\partial \tilde{v}} \right|\right|^2 = G/(\sqrt{G})^2 = 1, \text{ e}
    $$
    $$
    \tilde{F} = \langle \tilde{\sigma}_{\tilde{u}}, \tilde{\sigma}_{\tilde{v}} \rangle = \frac{\partial u}{\partial \tilde{u}}\frac{\partial v}{\partial \tilde{v}}\langle \sigma_u, \sigma_v \rangle = \frac{F}{\sqrt{EG}}.
    $$
    Observe que $F < \sqrt{EG}$ por Cauchy-Schwartz. Como esses mapas são
    suaves, podemos definir $\theta(\tilde{u},\tilde{v})$ suave entre 0 e
    $\pi$ de forma que $\cos(\theta) = F/\sqrt{EG}$. 

    Por fim, considere a transformação sugerida
    $$
    \tilde{u} = \frac{\hat{u} + \hat{v}}{2} \text{ e } \tilde{v} = \frac{\hat{u} - \hat{v}}{2}. 
    $$
    Assim a primeira forma fundamental é 
    $$
    \frac{1}{4}(d\hat{u} + d\hat{v})^2 + \frac{1}{2}\cos(\theta)(d\hat{u}^2 - d\hat{v}^2) + \frac{1}{4}(d\hat{u} - d\hat{v})^2 = \frac{1}{2}(1 + \cos(\theta))d\hat{u}^2 + \frac{1}{2}(1 - \cos(\theta))d\hat{v}^2.
    $$
    Com as propriedades trigonométrica, teremos que a primeira forma
    fundamental é 
    $$
    \cos^2(\theta/2)d\hat{u}^2 + \sin^2(\theta/2)d\hat{v}^2.
    $$

    

\end{solution}

\begin{exercise}[6.2.1]
    Pensando sobre como um cone circular pode ser "desembrulhado" em um plano,
    escreva uma isometria de 
    $$
    \sigma(u,v) = (u\cos(v), u\sin(v), u), u > 0, 0 < v < 2\pi,
    $$
    (um meio cone circular com uma reta removida) a um aberto no plano XY. 
\end{exercise}

\begin{solution}
    Vamos imaginar que abrimos o cone e colocamos no plano de forma espichada.
    Assim o terceiro componente deverá ser 0. A cada seção transversal, temos
    que os pontos tem norma $u\sqrt{2}$. Assim, no plano, essa será a
    distância deles à origem. Por fim, o ângulo de rotação vai ser diminuído
    para $v/\sqrt{2}$. Por isso, definimos 
    $$
    \tilde{\sigma}(u,v) = u\sqrt{2}\left(\cos(v/\sqrt{2}), \sin(v/\sqrt{2}), 0\right).
    $$
    Calculando as formas fundamentais de cada patch, veremos que são iguais,
    e, portanto, por \cite[Corolário 6.2.3]{pressley}, as superfícies são
    localmente isométricas. 
\end{solution}

\begin{exercise}
    Calcule a área do toro de revolução
\end{exercise}

\begin{solution}
    A parametrização do torus pode ser dada pelo exercício \cite[4.2.5]{pressley}
    $$
    \sigma(\theta, \phi) = ((a + b \cos \theta) \cos \phi, (a + b \cos \theta) \sin \phi, b \sin \theta).  
    $$
    para cada ponto, devemos escolher o intervalo de definição. Vamos
    considerar $R = (0,2\pi) \times (0,2\pi)$.
    $$
    \sigma_{\theta} = (-b\sin\theta\cos\phi, -b\sin\theta\sin\phi,b\cos\theta),
    $$
    $$
    \sigma_{\phi} = (a+b\cos\theta)(-\sin\phi, \cos\phi,0), \text{ e}
    $$
    $$
    E = b^2, F = 0, G = (a + b\cos\theta)^2.
    $$
    Portanto, 
    $$
    \mathcal{A}_{\sigma}(R) = \int_0^{2\pi}\int_0^{2\pi} (EG - \cancel{F^2})^{1/2} \, d\theta \, d\phi = b\int_0^{2\pi} \int_0^{2\pi} (a + b\cos\theta) \, d\theta \, d\phi = 4\pi^2 ab.
    $$
\end{solution}

\begin{thebibliography}{9}
     \bibitem{pressley} 
     Pressley, Andrew N. Elementary differential geometry. Springer Science \& Business Media, 2010.
\end{thebibliography}

\end{document}