\documentclass[a4paper,12pt]{article}

%%%%%%%%%%%%%%%%%%%%%%%%%%%%%%%%%%%%%%%%%%%%%%%%
% Packages
%%%%%%%%%%%%%%%%%%%%%%%%%%%%%%%%%%%%%%%%%%%%%%%%

\usepackage[right=2.5cm, left=2.5cm, top=2.5cm, bottom=2.5cm]{geometry} 
\usepackage[portuguese]{babel}
\usepackage[T1]{fontenc}
\usepackage[utf8]{inputenc}
\usepackage{url}
\usepackage{hyperref}
\Urlmuskip=0mu  plus 10mu

% no indentation
%\usepackage{setspace}
%\setlength{\parindent}{0in}

\usepackage{graphicx} 
\usepackage{float}
\usepackage{xcolor}

\usepackage{mathtools}
\usepackage{amssymb, amsthm}

% headers
\usepackage{fancyhdr}

%%%%%%%%%%%%%%%%%%%%%%%%%%%%%%%%%%%%%%%%%%%%%%%%
% Proper definitions
%%%%%%%%%%%%%%%%%%%%%%%%%%%%%%%%%%%%%%%%%%%%%%%%
\newcommand{\R}{\mathbb{R}}
\newcommand{\B}{\mathcal{B}}

\newtheoremstyle{exer}{}{}{\color{blue}}{}{\color{blue}\bfseries}{}{ }{}
\theoremstyle{exer}
\newtheorem{exercise}{Exercício}

\theoremstyle{definition}
\newtheorem{solution}{Solução}
\newtheorem{definition}{Definição}

\theoremstyle{plain}
\newtheorem{remark}{Observação}



%%%%%%%%%%%%%%%%%%%%%%%%%%%%%%%%%%%%%%%%%%%%%%%%
% Header (and Footer)
%%%%%%%%%%%%%%%%%%%%%%%%%%%%%%%%%%%%%%%%%%%%%%%%

\pagestyle{fancy} 
\fancyhf{}

\lhead{\footnotesize CS: Lista 7}
\rhead{\footnotesize Prof. Asla e Mon. Lucas} 
\cfoot{\footnotesize \thepage} 


\begin{document}

%%%%%%%%%%%%%%%%%%%%%%%%%%%%%%%%%%%%%%%%%%%%%%%%
% Title section of the document
%%%%%%%%%%%%%%%%%%%%%%%%%%%%%%%%%%%%%%%%%%%%%%%%

\thispagestyle{empty} 

\begin{tabular*}{0.95\textwidth}{l @{\extracolsep{\fill}} r} 
    {\large \bf Curvas e Superfícies 2021.1} &  \\
    Escola de Matemática Aplicada, Fundação Getulio Vargas &  \\
    Professora Asla Medeiros e Sá &  \\ 
    Monitor Lucas Machado Moschen & Entrega 07/06/2021\\
    \hline \\
\end{tabular*} 
\vspace*{0.3cm} 

\begin{center}
	{\Large \bf Lista 7}
	\vspace{2mm}
\end{center}  
\vspace{0.4cm}

\begin{exercise}[6.1.1]
    Calcule a primeira forma fundamental das seguintes superfícies: 
    \begin{enumerate}
        \item[(i)] $\sigma(u,v) = (\sinh(u)\sinh(v), \sinh(u)\cosh(v), \sinh(u))$.
        \item[(ii)] $\sigma(u,v) = (u-v,u+v,u^2+v^2)$.
        \item[(iii)] $\sigma(u,v) = (\cosh(u),\sinh(u),v)$.
        \item[(iv)] $\sigma(u,v) = (u,v,u^2+v^2)$.
        Que tipos de superfícies são estas? 
    \end{enumerate}
\end{exercise}

\begin{solution}

\end{solution}

\begin{exercise}[6.1.3]
    Seja $Edu^2 + 2Fdu dv + Gdv^2$ a primeira forma fundamental do patch
    $\sigma(u, v)$ da superfície $\mathcal{S}$. Mostre que, se $p$ é um
    ponto da imagem de $\sigma$ e $v$, $w \in T_p\mathcal{S}$, então 
    $$\langle v, w \rangle = Edu(v)du(w) + F(du(v)dv(w) + du(w)dv(v)) + Gdv(w)dv(w).$$
\end{exercise}

\begin{solution}

\end{solution}

\begin{exercise}[6.1.5]
    Mostre que as seguintes condições são equivalentes em um patch $\sigma(u,
    v)$ com primeira forma fundamental $Edu^2 + 2F dudv + Gdv^2$:
    \begin{enumerate}
        \item[(i)] $E_v = G_u = 0$.
        \item[(ii)] $\sigma_{uv}$ é paralelo ao vetor normal padrão $N$. 
        \item[(iii)] O lado oposto de qualquer quadrilátero formado por curvas
        paramétricas de $\sigma$ tem o mesmo comprimento (veja as observações após a Proposição 4.4.2).   
    \end{enumerate}
    Quando essas condições são satisfeitas, as curvas paramétricas de $\sigma$
    são ditas {\em Chebyshev net.} Mostra que, nesse caso, $\sigma$ tem uma
    parametrização $\tilde{\sigma}(\tilde{u}, \tilde{v})$ com a primeira forma
    fundamental 
    $$
    d\tilde{u}^2 + 2\cos(\theta)d\tilde{u}d\tilde{v} + d\tilde{v}^2,
    $$
    onde $\theta$ é uma função suave de $(\tilde{u}, \tilde{v})$. Mostra que
    $\theta$ é o ângulo entre as curvas paramétricas de $\tilde{\sigma}$.
    Mostre além que, se colocamos $\hat{u} = \tilde{u} + \tilde{v}, \hat{v} =
    \tilde{u} - \tilde{v}$, a reparametrização resultante
    $\hat{\sigma}(\hat{u}, \hat{v})$ de $\tilde{\sigma}(\tilde{u}, \tilde{v})$
    tem primeira forma fundamental 
    $$
    \cos^2(\omega)d\hat{u}^2 + \sin^2(\omega)d\hat{v}^2, 
    $$
    onde $\omega = \theta/2$. 
\end{exercise}

\begin{solution}

\end{solution}

\begin{exercise}[6.2.1]
    Pensando sobre como um cone circular pode ser "desembrulhado" em um plano,
    escreva uma isometria de 
    $$
    \sigma(u,v) = (u\cos(v), u\sin(v), u), u > 0, 0 < v < 2\pi,
    $$
    (um meio cone circular com uma reta removida) a um aberto no plano XY. 
\end{exercise}

\begin{solution}

\end{solution}

\begin{exercise}
    Calcule a área do toro de revolução
\end{exercise}

\begin{solution}

\end{solution}

% \begin{thebibliography}{9}
%     \bibitem{pressley} 
%     Pressley, Andrew N. Elementary differential geometry. Springer Science \& Business Media, 2010.
% \end{thebibliography}

\end{document}