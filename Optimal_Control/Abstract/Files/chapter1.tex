\subsection*{Exemplo 1}

Como motivação apresentada, consideramos duas equações que representam a variação do peso da parte vegetativa e reprodutiva, respectivamente. Nesse caso, o controle é a fração da fotossíntese destinada para a parte vegetativa. Nosso objetivo, nesse caso, é maximixar: 
\begin{equation*}
F(x,u,t) := \int_0^T \ln(x_2(t))dt, s.a. 
\end{equation*} 
\begin{equation*}
\begin{cases}
x_1'(t) = u(t)x_1(t) \\
x_2'(t) = (1 - u(t))x_2(t) \\
0 \leq u(t) \leq 1
\end{cases}    
\end{equation*}

Desta maneira, em um sistema de controle há variáveis de estado, funções de controle, que afetam a dinâmica do sistema, e o funcional objetivo, que funciona como o cálculo do lucro. 

\subsection{Definições Importantes}
\begin{enumerate}
    \item \textbf{Continuidade por partes:} Se uma função é contínua em cada ponto em que é definida, exceto em uma quantidade finita e é igual a seu limite à esquerda ou à direita me cada ponto (não permite pontos deslocados). 
    \item \textbf{Diferenciável por partes:} Função contínua que é diferenciável em cada ponto que é definida, exceto uma finidade.
    \item \textbf{Côncava: } Se $\forall 0 \leq \alpha \leq 1$ e $\forall a \leq t_1,t_2 \leq b$, $\alpha k(t_1) + (1 - \alpha)k(t_2) \leq k(\alpha t_1 + (1 - \alpha)t_2)$.
    \item \textbf{Lipschitz: } $|k(t_1) - k(t_2)| \leq c|t_1 - t_2|$.
    \item \textbf{Teorema do Valor Médio}
    \item \textbf{Teorema da Convergência Dominante: } Considere uma sequência $\{f_n\}$ dominada por uma função Lebesgue integrável $g$. Suponha que sequência converge ponto a ponto para uma função $f$. Então $f$ é integrável e $\lim_{n \to \infty} \int_S f_n d\mu = \int_S f d\mu$.
\end{enumerate}

\subsection*{Exercício 1}
Se $k: I \to \mathbb{R}$ é diferenciável por partes em um intervalo limitado, $k$ é Lipschitz. 

\textbf{Prova:} Seja $P = \{p_0, ..., p_n\}$ pontos em que $k$ não é diferenciável. Pelo teorema do valor médio, $\forall i, \exists x_i^0 $, tal que $k(p_i) - k(p_{i-1}) = k'(x_i^0)(p_i - p_{i-1})$. Como $k'$ é contínua em cada intervalo $[p_{i-1},p_i]$, basta tomar o valor máximo da derivada nesse intervalo. Depois disso, basta tomar o maior valor entre todas as derivadas e a desigualdade de Lipschitz é satisfeita. 

\subsection{Problema} 

Considere $x'(t) = g(t,x(t),u(t)) \to u(t) \mapsto x(u)$. Queremos, então, dado um funcional $ J(u) := \int_{t_0}^{t_1} (f(t,x(t),u(t))dt, x(t_0) = x_0$, maximizá-lo.  $u(t)$ é contínua por partes. 

\textbf{Função Adjunta:} proposta similar aos multiplicadores de Lagrange. $\lambda : [t_0,t_1] \to \mathbb{R}$ é diferenciável por partes e deve satisfazer algumas condições. 

Para isso, assumimos a existência $u^*$ e $x^*$. Nesse caso, $J(u) \leq J(u^*) < \infty$. Tome $u^{\epsilon} = u^{*}(t) + \epsilon h(t), x^{\epsilon}(t_0) = x_0 \implies \lim_{\epsilon\to 0} u^{\epsilon} = u^{*}$. $x^{\epsilon}$ é o estado associado ao controle. Como a função $g$ é continuamente diferenciável, $x^{\epsilon} \to x^{*}$. Assim, a sua derivada em $\epsilon = 0$ existe. Utilizo, pelo TFC, que $\int_{t_0}^{t_1} \frac{d}{dt}(\lambda (t)x^{\epsilon}(t))dt = \lambda(t_1)x^{\epsilon}(t_1) - \lambda(t_0)x^{\epsilon}(t_0)$. Desta maneira, utilizo que $J(u^{\epsilon}) = \int_{t_0}^{t_1} f(t,x^{\epsilon}(t),u^{\epsilon}(t))dt - \lambda(t_1)x^{\epsilon}(t_1) + \lambda(t_0)x^{\epsilon}(t_0)$. Sabemos que $lim_{\epsilon \to 0} \frac{J(u^{\epsilon}) - J(u^{*})}{\epsilon} = 0$, pois $J(u^*)$ é máximo. Desta maneira, 
\begin{align} \label{eq1}
0 = \frac{d}{d\epsilon} J(u^{\epsilon})_{\e = 0} = \int_{t_0}^{t_1} [(f_x + \lambda(t)g_x & + \\ 
 + \lambda '(t))\frac{\partial x^{\e}}{\partial\e}(t)_{\e = 0} + (f_u + \lambda(t)g_u)h(t)]dt - \lambda(t_1)\frac{\partial x^{\e}}{\e}(t_1)_{\e = 0}
\end{align}

Note que isso pode ser feito pelo Teorema da Convergência Dominante, pois podemos mover o limite (derivada) para dentro da integral (intervalo compacto e integrando é diferenciável po partes).  Para que ocorra a igualdade citada acima, definimos $H(t,x,u,\lambda) := f(t,x,u) + \lambda g(t,x,u)$ e:
\begin{equation}
\begin{cases}
    \frac{\partial H}{\partial u}_{u = u^{*}} = 0 \\
    \frac{\partial H}{\partial x}_{x = x^*} = - \lambda ' \\
    \frac{\partial H}{\partial \lambda} = x' \\
    \lambda(t_1) = 0 
\end{cases}    
\end{equation}

\subsection{Princípio Máximo de Pontryagin}

Se $u^*$ e $x^*$ são controle ótimo, então existe $\lambda (t)$ diferenciável por partes tal que a função $H$, como definida anteriormente, pode ser maximizada em $u^*(t)$. A demonstração é mais simples para o caso de $f$ e $g$ côncavas em $u$ e $\lambda (t) \geq 0$. A segunda derivada do Hamiltoniano indica o tipo de problema: Se for negativa, é um problema de maximização. 
\textbf{Observação:} A condição de maximizar $H$ não sempre implica que $H_u = 0$.  

\subsection{Exercício 1.6 - Efeito Alle}

Nesse efeito, consideramos um valor mínimo. O crescimento $x'(t) = rx(t)(\frac{x(t)}{x_{min}} - 1)(1 - \frac{x(t)}{x_{max}})$

